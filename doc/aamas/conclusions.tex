\section{Conclusions and Future Work}
\label{sec:conclusions}

% Contributions
% - new scheme for generating options
We have devised a new scheme to generate options based on small world
network model. The options generated satisfy an intuitive criteria, that
the subtasks learnt should be easily composed to solve any other task.
The options greatly improve the connectivity properties of the domain,
without leading to a state space blow up. Finally, they are interesting
from a theoretical perspective, as they require only a logarithmic
number of decisions required in a learning task.

% - absolutely model-free
Experiments run on standard domains show significantly faster learning
rates using small world options. At the same time, we have shown that
learning small world options can be cheaper than learning bottleneck
options, using a natural algorithm that extracts options from a handful
of tasks it has solved. Another advantage of the scheme is that is does
not require a model of the MDP. 

% Further work
% - dynamically add/remove options
% - figuring out r
As future work, we would like to characterise what the exponent $r$
should be in a general domain. Given the ease with which options can be
discovered, it would be interesting to experiment with a dynamic scheme
that adds options on the fly, while solving tasks.


\begin{abstract}

Understanding how we are able to perform a diverse set of complex tasks
has been a central question for the Artificial Intelligence community.
One popular approach is to use temporal abstraction as a framework to
capture the notion of subtasks. However, this transfers the problem to
finding the right subtasks, which is still an open problem. Existing
approaches for subtask generation require too much knowledge of the
environment, and the abstractions they create can overwhelm the agent.
We propose a simple algorithm inspired by small world networks to learn
subtasks while solving a task that requires virtually no information of
the environment. Additionally, we show that the subtasks we learn can be
easily composed by the agent to solve any other task; more formally, we
prove that any task can be solved using only a logarithmic combination
of these subtasks and primitive actions. Experimental results show that
the subtasks we generate outperform other popular subtask generation
schemes on standard domains. 

%\draft{What is our contribution?}
% Relevance to lifelong learning?

\end{abstract}
